
\title{Unternehmensorientierung - Lernen und Wissen in Unternehmen}

\author{
  Benjamin Vetter, Andreas Krohn \\
  firstname.lastname@haw-hamburg.de \\
  Department of Comupter Science \\
  Hamburg University of Applied Science (HAW) \\
  Berliner Tor 7, 20099, Hamburg \\
  benjamin.vetter@haw-hamburg.de \\
}

\date{\today}

\documentclass[12pt]{article}

\usepackage[utf8]{inputenc}
\usepackage{hyperref}

\begin{document}
\maketitle

\section{Abstract (deutsch)}

Der Begriff ,,Lernen'' kann als die aus Erfahrung resultierende, dauerhafte
Veränderung des Verhaltens eines Individuums definiert werden
\cite{Lefrancois:2006}. Den Prozess des Lernens versuchen dabei zahlreiche
klassische Lerntheorien, wie Behaviorismus, Koginitivismus, Konstruktivismus
oder Konnektivismus zu erklären. Organisationale Lerntheorien wiederrum
versuchen den Lernprozess in Unternehmen zu erklären und setzen z.T. auf den
klassischen Lerntheorien auf. Dabei ist organisationales Lernen mehr als die
bloße Summe des individuellen Lernens der Mitarbeiter eines Unternehmens. So
geht es darum, basierend auf individuellem und Gruppenlernen Lernprozesse der
Mitarbeiter zu initiieren, zu stimulieren und zu steuern \cite{Franken:2002}.
Hierzu sollten Unternehmen eine offene, konstruktive Lern- und
Diskussionskultur schaffen \cite{culture}. Dabei sind jedoch zahlreiche
Hindernisse, wie undeterministische Lernkurven, Kommunikationsprobleme oder
seleketive Warhnehmung zu überwinden.  Um den Lernprozess in Unternehmen besser
zu verstehen, bedient man sich der Analogie, das Unternehmen als Individuum zu
definieren, sodass die Unternehmenskultur der Persönlichkeit des Individuums
entspricht und das unternehmerische Handeln dem bewussten Tun. Ein Individuum
kann demzufolge bspw. lernen, indem die Konsequenzen des bewussten Handelns
bspw. bei Diskrepanzen bzgl. eigener Erwartungen Lernprozesse initiieren, die
die eigenen Handlungstheorien modifizieren \cite{Pawlowsky:1992}. Das
Individuum verändert sein Verhalten, sodass innerhalb der Unternehmensumwelt
Probleme bzgl. unternehmerischen Zielen besser gelöst werden können. Die
organisationalen Lerntheorien von Richard Cyert und James March, Chris Argyris
und Donald Schön, Peter Senge, sowie Ikujiro Nonaka und Hirotaka Takeuchi
vertreten unterschiedliche, auch philosophische Richtungen und stellen
verschieden hohe Anforderungen an Mitarbeiter und Management. Als einizge,
praktische Handlungskriterien sind sie jedoch nicht zu verwenden, zumal der
Lernprozess aufgrund unterschiedlichen Unternehmenskulturen und -umwelten
betriebsspezifisch angepasst werden muss.

%Die organisationale Lerntheorie von Richard Cyert und James March hat ihren
%Ursprung in den 1950er Jahren. Das Unternehmen wird als adaptvies, passives
%System verstanden, das lernt indem die eigenen Ziele und Regeln den
%Anforderungen der Unternehmensumwelt angepasst werden.  Organisationales Lernen
%findet hierbei primär bei Krisen statt. Es müssen akute Probleme existieren,
%damit nach Handlungsalternativen gesucht wird, sodass sich die
%Unternehmensumwelt, überspitzt formuliert, das Unternehmen gemäß Behaviorismus
%kondiditioniert, was aber nur langsam und in Zyklen geschieht.  Die Lerntheorie
%von Chris Argyris und Donald Schön verfolgt sowohl Ansätze des Kognitivismus
%als auch des Konstruktivismus. So wird gemäß Kognitivismus davon ausgegangen,
%dass ein Unternehmen in der Lage ist, selbständig kognitive Lernprozesse zu
%gestalten \cite{Franken:2002}. Gemäß Konstruktivismus konstruiert das
%Unternehmen seine eigene Wirklichkeit erfordert aber ständige Reflektion ob das
%Wahrgenommene, mit der eigenen, konstruieren Realität übereinstimmt und stellt
%damit hohe Anforderungen an Mitarbeiter und Management, damit ein möglichst
%hohes Lernniveau erreicht wird. Im Zentrum der Japaner Nonaka und Takeuchi
%steht die Wissensgenerierung, da westliche Unternehmen zu stark auf die
%Wissensverarbeitung fokussiert seien. Dabei untersuchen Sie, wie
%organisationales Lernen mittels Sozialisation, Extern- und Internalisierung,
%sowie Kombination möglich ist. Dabei setzt auch diese Theorie Anforderungen an
%Mitarbeiter und Management.

Ergebnis des Lernens soll ``Wissen'' sein. Zu diesem Begriff existieren
vielfältige Definitionen \cite{Schilcher:2006}. Im Unternehmenskontext ist die
Klassifikation nach Zugänglichkeit (implizit und explizit) sowie nach personeller
Bindung (individuell und kollektiv) nach Nonaka und Takeuchi \cite{Nonaka:1997}
sinnvoll. Erfolgreiches Wissensmanagement soll Unternehmen auch das individuelle,
implizite Wissen der Mitarbeiter erschließen und erweitern helfen um damit Kosten
zu senken und Innovation zu befeuern.
% TODO westl. Unternehmen vs. Japan?
Mit den ``Bausteinen des Wissensmanagements'' definieren Gilbert J. B. Probst, 
Steffen Raub und Kai Romhardt \cite{Probst:2006} Aspekte des Wissensmanagements.
%Sie gehen jedoch von einer nicht näher erleuterten Möglichkeit zur Messung von
%Wissensfortschritten aus und liefern auch sonst wenig praktisch umsetzbares.
Swetlana Franken \cite{Franken:2007} stellt den Mitarbeiter in den Mittelpunkt.
Passende Rahmenbedingungen sollen die Voraussetzungen für freiwilliges Engagement
schaffen.

\section{Abstract (english)}

Learning can be defined as a permanent change of behaviour of an individual,
induced by experience \cite{Lefrancois:2006}. Classical learning theories, like
the behaviorism, cognitivism, constructivism or connectivism try to explain how
the process of learning works in detail. Furthermore, organizational learning
theories, while some of them are based on these classical learning theories,
explain how learning is possible in companies, such that a group-wise, a
collective learning is achieved. Organizational learning is more than simply
the sum the individually learning employees. More specifically, organizational
learning is all about initiating, stimulating and managing processes that
enable the employees and, in particular, groups of employees to learn
\cite{Franken:2002}. To achieve organizational learning, companies need to
establish an open minded learning culture, supporting constructive debates
\cite{culture}. However, many obstacles and barriers, like varying learning
curves, personal conflicts and selective perception must be challenged.  To
better undestand organizational learning processes, companies can be treated as
individuals, such that the organizational culture corresponds to the
individual's personality and the corporate policies correspond to its concious
behaviour. According to this analogy, an organization is able to learn, because
the consequences of its concious behaviour can potentially stimulate an
adoption of its rules and behavioral theories, e.g., by recognizing
differences between expected and actual consequences, such that the individual
learns \cite{Pawlowsky:1992}. The organization changes its behaviour to better
solve problems within its environment in terms of economical goals. The
organizational theories of Richard Cyert and James March, Chris Argyris and
Donald Schön, Peter Senge, as well as Ikujiro Nonaka und Hirotaka Takeuchi
follow different, even philosophical routes and make varying demands on the
employees as well as the organization's management. However, they can and
should not be used as well as seen as the only criterion, since the process of
organizational learning must be individually adapted to the company's culture
and environment.

\nocite{*}
\bibliographystyle{plain}
\bibliography{abstract}

\end{document}

